\documentclass[10pt,sigconf,anonymous]{acmart}
\synctex=1
\graphicspath{{figures/}{doc/paper/}}
\usepackage[l2tabu,orthodox]{nag}
\usepackage[utf8x]{inputenc}
\usepackage[british]{babel}
\usepackage{microtype}
\usepackage[caption=false]{subfig}
\usepackage{graphicx}
\usepackage{url}
\usepackage{color}

\frenchspacing
\uchyph=0

\newcommand{\todo}[1]{\textbf{\textcolor{red}{To do: #1}}}
\newcommand{\idea}[1]{{\textcolor{blue}{#1}}}

%==================================================================================================
\begin{document}
\title{Does TCP's New Congestion Window Validation Improve HTTP Adaptive Streaming Performance?}

\author{Mihail Yanev}
    \orcid{0000-0002-9814-1017}
    \affiliation{
      \institution{University of Glasgow}
      \streetaddress{School of Computing Science}
      \city{Glasgow}
      \postcode{G12 8QQ}
      \country{UK}
    }
    \email{m.yanev.1@research.gla.ac.uk}


\author{Stephen McQuistin}
    \orcid{0000-0002-0616-2532}
    \affiliation{
    \institution{University of Glasgow}
    \streetaddress{School of Computing Science}
    \city{Glasgow}
    \postcode{G12 8QQ}
    \country{UK}
    }
    \email{}

\author{Colin Perkins}
  \orcid{0000-0002-3404-8964}
  \affiliation{
    \institution{University of Glasgow}
    \streetaddress{School of Computing Science}
    \city{Glasgow}
    \postcode{G12 8QQ}
    \country{UK}
  }
  \email{csp@csperkins.org}


\acmYear{2022}
\copyrightyear{2022}
\setcopyright{acmcopyright}
\acmConference{NOSSDAV '22}{June 17, 2022}{Athlone, Ireland}
\acmPrice{15.00}
% \acmDOI{10.1145/3284850.3284856}
% \acmISBN{978-1-4503-6082-1/18/12}

%==================================================================================================
\begin{abstract}
  % Four sentences:
  %  - State the problem
  %  - Say why it's an interesting problem
  %  - Say what your solution achieves
  %  - Say what follows from your solution

HTTP adaptive streaming video flows exhibit on-off behaviour, with frequent idle periods, which can interact poorly with TCP's congestion control algorithms. New congestion window validation (New CWV) modifies TCP to allow senders to restart more quickly after certain idle periods. While previous work has shown that New CWV can improve \emph{transport} performance for streaming video, it remains to demonstrate that this translates to improved \emph{application} level performance, in terms of playback stability. In this paper, we show that enabling New CWV can reduce video re-buffering events by up to 4\%, and limit representation switches by 12\%, without any changes to existing rate adaptation algorithms.

\end{abstract}
\maketitle
%==================================================================================================
\section{Introduction}
\label{sec:introduction}

% A good paper introduction is fairly formulaic. If you follow a simple set
% of rules, you can write a very good introduction. The following outline can
% be varied. For example, you can use two paragraphs instead of one, or you
% can place more emphasis on one aspect of the intro than another. But in all
% cases, all of the points below need to be covered in an introduction, and
% in most papers, you don't need to cover anything more in an introduction.
%

Video streaming over HTTP is commonplace, and comprises the majority of Internet traffic~\cite{Sandvine-2019-global-internet-report}. Performance of HTTP adaptive streaming is generally good, and gives a high-quality user experience~\todo{\cite{..}}.
  
There are, however, some scenarios where HTTP adaptive streaming performs poorly~\cite{Spiteri-2016-BOLA,Kua-2017-a-survey-rate-adaptation-dash}. In particular, the interaction between the on-off traffic patterns generated by chunked streaming applications and TCP congestion control algorithms can reduce the performance of throughput-based video rate adaptation schemes~\cite{Akhshabi-2012-http-adaptive-players-compete,Stohr-2017-where-are-the-sweet-spots-maci}. In some cases, this is due to TCP's congestion window validation (CWV)~\cite{rfc2861-2000-padhye-congestion-window-validation} algorithm, which, while preventing TCP clients from sending using stale knowledge of the network, has been shown to negatively impact the throughput of rate-limited applications~\cite{Nazir-2014-performance-evaluation-congestion-window-validation-dash-newcwv}, including HTTP adaptive streaming. New congestion window validation (New CWV)~\cite{rfc7661-2015-fairhurst-new-cwnd-validation} has been proposed to address this. Prior work~\cite{Nazir-2014-performance-evaluation-congestion-window-validation-dash-newcwv} has demonstrated that New CWV has the desired \emph{transport} layer impact, but it remains to show that this translates to improved quality of experience (QoE) performance at the \emph{application} layer. This is not guaranteed, given the complexity that exists at both layers, and that results from their interaction. For example, large discrepancies between the video's bandwidth requirements and the available link capacity, or the requirement for stable, long-lived connections in modern streaming video players (e.g., dash.js~\cite{online-dashjs}), can influence rate adaptation~\cite{Spiteri-2019-from-theory-to-practice-sabre}.

% Paragraph 1: Motivation. At a high level, what is the problem area you
% are working in and why is it important? It is important to set the larger
% context here. Why is the problem of interest and importance to the larger
% community?

% Video playback stability is one of the key components contributing to user experience. Adaptive algorithms residing at the client are responsible for delivering high user experience. Some current state-of-the-art solutions use network throughput as a bitrate selection heuristic.

% Paragraph 2: What is the specific problem considered in this paper? This
% paragraph narrows down the topic area of the paper. In the first
% paragraph you have established general context and importance. Here you
% establish specific context and background.

% The adaptive algorithms face the difficult task of delivering video while picking its ``best'' representations at any given point in time. This is since these algorithms need to simultaneously fulfil multiple criteria including: high stability, high quality, and low video stalls. These criteria can often conflict one another. For example, an algorithm always picking the lowest available representation will achieve low video stall time and high stability (since representations will not fluctuate), however, it will also display lower video quality, whereas perhaps higher one was possible. In a dynamic network, achieving a high index on one of these criterion also comes at the cost of observing a lower index in another. Modern adaptive algorithms try to balance that trade-off. They use either the pre-buffered capacity at the client or the immediate network state as a heuristic to aid them in the decision process. The former algorithms are known as buffer and the latter as throughput based. 


% One way to influence the client's throughput calculations is to change the transport. Video over HTTP has to transfer many video chunks over the same connection. The time when between each such transfers, when no information is exchanged, is known as an ``idle period''. RFC 5861 recommends that during such periods TCP should adjust its congestion window back to its initial window value and re-enter slow start \cite{rfc5681-congeston-control}. New Congestion Window Validation \cite{rfc7661-2015-fairhurst-new-\emph{cwnd}-validation} proposes a way for the congestion window value to remain unchanged during idle periods and connections using it do not need to re-enter slow start. These two approaches have different impact on the client's throughput calculations.


% However, different client throughput estimations does not necessarily mean improved video QoE. Video QoE is a complex variable that is a combination of different application metrics, e.g., average bitrate, average bitrate oscillation, and rebuffer ratio. As discussed earlier, improving one could come at the cost of worsening others. Therefore, the implications of a transport change are unclear with respect to the video's QoE or smoothness.


% Paragraph 3: "In this paper, we show that...". This is the key paragraph
% in the introduction - you summarize, in one paragraph, what are the main
% contributions of your paper, given the context established in paragraphs
% 1 and 2. What's the general approach taken? Why are the specific results
% significant? The story is not what you did, but rather:
%  - what you show, new ideas, new insights
%  - why interesting, important?
% State your contributions: these drive the entire paper.  Contributions
% should be refutable claims, not vague generic statements.

In this paper, we investigate whether enabling New CWV improves video playback stability, and more generally, improves video QoE. To test our hypothesis, we compare two video streams using TCP New Reno, one with CWV and with New CWV. We collect standard video performance metrics, including bit-rate oscillation, and stall time, to measure stability and QoE. Further, to quantify the impact of New CWV with respect to the inferred network state at the client, we also record the immediate and smoothened client's current link capacity estimations for each delivered video chunk.% We carry our experiments using TCP New Reno, as previous study found that it is widely used for video delivery \cite{Mishra-2019-the-great-internet-tcp-congestion-control-census} and because New CWV is agnostic to the used congestion control algorithm.

In particular, we make the following contributions: (i) an implementation of New CWV for Linux (kernel version 5.4)~\footnote{The code used in this paper will be made available with the camera-ready version.}; (ii) a testbed setup for evaluating New CWV's application layer impact; and (iii) results that demonstrate that New CWV improves video stability, with an \todo{x}\% reduction in bit rate switches, and a \todo{y}\% reduction in rebuffering time.

% Paragraph 4: What are the differences between your work, and what others
% have done? Keep this at a high level, as you can refer to future sections
% where specific details and differences will be given, but it is important
% for the reader to know what is new about this work compared to other work
% in the area.

To the best of our knowledge, this is the first paper that studies New CWV's application layer impact. Nazir et al. \cite{Nazir-2014-performance-evaluation-congestion-window-validation-dash-newcwv} demonstrated New CWV's effect on the transport layer: we validate their results in \S\ref{sec:transport-impact}. There has been a large amount of work that has proposed new application layer rate adaptation algorithms~\cite{Mok-2012-qdash,Huang-2015-A-buffer-based-approach-to-rate-adaptation-bba, Yin-2015-a-control-theoritic-approach}. In contrast, we only change the transport algorithm and leave the application as is, studying the transport's impact on the application. Improving performance via transport layer modifications could allow for simpler rate adaptation algorithms at the application layer. %Another proposal considers replacing the transport protocol from TCP to QUIC \cite{Bhat-2017-not-so-quic} but find that only doing this brings little to no benefit. 

% Paragraph 5: "We structure the remainder of this paper as follows." Give
% the reader a road-map for the rest of the paper. Try to avoid redundant
% phrasing, "In Section 2, In section 3, ..., In Section 4, ... ", etc.

We structure the remainder of this paper as follows. In Section~\ref{sec:background}, we introduce TCP congestion window validation, including its limitations with respect to HTTP adaptive video flows, before describing New CWV. Section~\ref{sec:evaluation} describes our experimental setup, and the transport and application layer impact of enabling New CWV. Section~\ref{sec:related} describes related work, and Section~\ref{sec:conclusion} concludes.

%==================================================================================================
\section{Congestion Window Validation}
\label{sec:background}

\begin{figure}
  \centering
    \subfloat[CWV]{
      \includegraphics[width=.23\textwidth]{figures/cwv.pdf}
      \label{fig:cwv}
    }
    \subfloat[New CWV]{
      \includegraphics[width=.23\textwidth,]{figures/new_cwv.pdf}
      \label{fig:newcwv}
    }
    \caption{Illustration of \emph{cwnd} growth following an idle period \todo{The abstract says that New CWV allows senders to restart more quickly: that isn't obvious from this illustration -- maybe needs more in the text/diagram to make it clear that the peak is made up lost segments -- connect better w/ Fig 2}}
    \label{fig:cwnd-growth-after-idle}
\end{figure}

In HTTP adaptive streaming, a server provides pre-encoded video chunks in different representations, each encoded at multiple bit rates, while the client, using a rate adaptation algorithm, determines the best representation to request at any given time. The goal of the client is to maximise QoE within the network's capacity. This can be a challenge since different, often contradictory, QoE heuristics need be considered simultaneously~\cite{Seufert-2015-A-Survey-on-QoE-Dash}. 
% Furthermore, since HTTP streaming uses mainly TCP to transfer the chunks, the bit rate selection's algorithm complexity rises.

Throughput-based rate adaptation algorithms for HTTP adaptive streaming use an estimate of the current network conditions to determine the representation that should be requested. These algorithms require a stable and accurate throughput estimate in order to perform well. However, the interaction between the on-off traffic pattern of streaming video and TCP's congestion control algorithm can lead to significant fluctuations in throughput, impacting the performance of throughput-based rate adaptation algorithms.

%Previous work has identified that the interaction between TCP and HTTP can lead to poor link utilisation and perceived quality of experience for HTTP adaptive connections \cite{Bae-2015-why-is-http-streaming-hard,Esteban-2012-Interactions-HTTP-TCP}. One impeding performance aspect of the interaction between the two layers is related to the On-Off nature of the video traffic.

In particular, during the idle periods in video transmission between chunks, the TCP congestion controller's knowledge of the network capacity becomes stale. To avoid sending with a possibly unrepresentative congestion window, the congestion window validation~\cite{rfc2861-2000-padhye-congestion-window-validation} (CWV) algorithm resets the TCP congestion window (\emph{cwnd}) to its initial value and forces the connection to re-enter slow-start after an idle period. Figure~\ref{fig:cwv} illustrates the behaviour of CWV following an idle period. CWV has become standard practice~\cite{rfc5681-congeston-control}, and is enabled by default in the latest stable Linux kernel (5.4).

While this approach has been employed and works for bulk, network-limited applications. It is unsuitable for rate-limited applications, including HTTP adaptive streaming~\cite{Esteban-2012-Interactions-HTTP-TCP}. To address this, new congestion window validation~\cite{rfc7661-2015-fairhurst-new-cwnd-validation} has been proposed. One of the modifications in New CWV is that rather than relying on slow-start until packet loss to re-discover an appropriate \emph{cwnd} value after an idle period, New CWV preserves the \emph{cwnd} before the idle period as its slow-start threshold (\emph{ssthresh}), i.e., it uses that value to later exit the slow-start phase. Figure~\ref{fig:newcwv} shows the growth of \emph{cwnd} following an idle period under New CWV.

\begin{figure}[t!]
  \centering
  \subfloat[CWV]{
    \includegraphics[width=.45\textwidth, keepaspectratio]{figures/lost_packets_vreno.pdf}
    \label{fig:transmission-after-idle-reno}
  }
  \\
    \subfloat[New CWV]{
      \includegraphics[width=.45\textwidth, keepaspectratio]{figures/lost_packets_newcwv.pdf}
      \label{fig:transmission-after-idle-newcwv}
    }
    \caption{Resumption after an idle period \todo{X-axis: ``Time since restart (s)``} \todo{Times} \todo{Highlight congestion control phases}}
    \label{fig:transmission-after-idle}
\end{figure}

To evaluate the performance of New CWV, we have implemented the algorithm within the Linux kernel. \todo{Some numbers}. We use this implementation to better illustrated the impact of New CWV on flows restarting after an idle period, as shown in Figure~\ref{fig:transmission-after-idle}.
As shown, the connection using New CWV uses the previously set \emph{ssthresh} value and leaves slow-start early. This results in New CWV connections not experiencing any packet loss (Figure~\ref{fig:transmission-after-idle-newcwv}), after reaching their set \emph{ssthresh} value in the third flight of packets after restarting. In contrast, if the same connection used CWV, the senders would not have preserved the \emph{ssthresh} value that way and would rely on loss to exit slow-start, as seen at the end of the third and fourth flights of packets in Figure~\ref{fig:transmission-after-idle-reno}. Overall, New CWV results in fewer lost packets, and returns to its previous sending rate without overshoot after loss, giving more predictable transmission.

New CWV has previously been shown to improve the \emph{transport} layer performance of rate-limited applications when compared with CWV~\cite{Nazir-2014-performance-evaluation-congestion-window-validation-dash-newcwv}, and our implementation and the results we have presented here validate that. It remains to show how this translates into \emph{application} layer performance, particularly since this is not guaranteed~\cite{Spiteri-2016-BOLA}. In Section~\ref{sec:evaluation}, we first validate the results of Nazir et al.~\cite{Nazir-2014-performance-evaluation-congestion-window-validation-dash-newcwv}, before testing our hypothesis that New CWV will enable applications to obtain more consistent throughput estimates, and, in turn, improve the stability of throughput-based rate adaptation algorithms.

\todo{Make it clear that this discussion is about transport performance}
\todo{Restructure: discussion about CWV (and Fig 2a) first, then New CWV (and Fig 2b)}

%==================================================================================================
\section{Evaluating New CWV for Video}
\label{sec:evaluation}

\begin{figure}
  \centering
  \includegraphics[width=.5\textwidth]{figures/setup.pdf}
  \caption{Experimental Setup \todo{Times}}
  \label{fig:experimental-setup}
\end{figure}

We first describe our experimental setup (\S\ref{sec:experimental-setup}), which we use to investigate the transport layer impact, verify whether New CWV connections obtain more consistent throughput estimates (\S\ref{sec:transport-impact}), and later to also investigate the application impact, and more specifically, the difference on video QoE that New CWV connections observe (\S\ref{sec:QoE-impact}). Finally, we summarise our findings (\S\ref{sec:summary}).

%--------------------------------------------------------------------------------------------------
\subsection{Experimental Setup}
\label{sec:experimental-setup}

% Overview
Our evaluation testbed consists of a network emulated in Mininet, running on Ubuntu 20.04, as shown in Figure \ref{fig:experimental-setup}. Both the server and its clients use TCP New Reno, widely used for video delivery~\cite{Mishra-2019-the-great-internet-tcp-congestion-control-census}. In addition, both are running a modified Linux Kernel (5.4.0). The modifications include a version of New CWV ported to that kernel, alongside RFC 3339~\cite{rfc3339-precise-timestamps} compliant timestamps, to enable better event tracking with higher timing precision. \texttt{tcpdump} is used to allow network activity to be reconstructed, which we use to study the packet loss.

% Server
The server uses \texttt{nginx} (version 1.18) with HTTP/2 delivery enabled. The server provides three representations of Big Buck Bunny~\cite{online-bbb}, encoded at 480p (requiring bandwidth of 0.44Mbps), 720p (2.64Mbps), and 1080p (4.82Mbps). Each representation is provided in chunks that are 3 seconds in duration.

 %  Client
Each client uses Firefox (version 91) with the dash.js (version 4.0.0) player. While the current state-of-the-art rate adaptation algorithm is DYNAMIC~\cite{Spiteri-2019-from-theory-to-practice-sabre}, we opt to use the throughput algorithm for our experiments. DYNAMIC combines the throughput algorithm with an enhanced version of the BOLA algorithm~\cite{Spiteri-2016-BOLA}. By using the throughput algorithm, our evaluations focus on the impact that the transport layer has on throughput estimation at the application layer. Our findings are applicable to the DYNAMIC algorithm, especially in scenarios where the throughput component of the algorithm is used (e.g., in low-latency, live streaming applications).

The network is configured with a bottleneck RTT of 40ms, a reasonable value to emulate connections within a country or region. The routers' queues are sized to the bandwidth delay product. Three different bandwidth profiles are evaluated, representing DSLv2 (10Mbps), FTTC (50Mbps), and FTTP (145Mbps) links; e.g., as are typical in the UK~\cite{online-ofcom-report}. Below we show mostly the results for the DSLv2 and FTTC links as the number of clients and the video representation demands were always unable to saturate an FTTP link's capacity. However, as higher resolution video and other network-heavy operations such as virtual reality environments become more available we expect these issues to translate to the links with higher capacity (i.e. FTTP). \todo{There are no results for FTTP reported below: would remove from the description here.}

To evaluate the impact of congestion and competing flows, each simulation was run with multiple clients (1, 2, 3, and 5 clients) simultaneously requesting video. Finally, to reduce noise, we ran each combination of CWV or New CWV, number of clients, and link type, 10 times before reporting the average results. The results presented includes data accumulated from 240 simulations (\emph{2 algorithms $\times$ 3 link types $\times$ 4 client variations $\times$ 10 repetitions}). 

During each run we collect the client's bandwidth estimations. Additionally, to evaluate the video QoE impact we collect information to report the rebuffer ratio and the bitrate switch frequency distribution.

%--------------------------------------------------------------------------------------------------
\subsection{Impact on Transport Performance} 
\label{sec:transport-impact}

New CWV alters TCP's \emph{cwnd} sizing behaviour, allowing it to recover more quickly after an idle period in an active TCP connection. As shown in Figures~\ref{fig:newcwv} and \ref{fig:transmission-after-idle-newcwv}, New CWV avoids the packet loss associated with CWV, and we therefore expect clients to report more stable available link bandwidth estimates. 

To evaluate this hypothesis, we collect the client's ``instantaneous'' and ``smoothed'' bandwidth estimates. The ``instantaneous'' estimate is obtained by dividing the size of the chunk, in bytes, by the time taken to download it. The ``smoothed'' estimate takes the ``instantaneous'' estimate, but combines it with other factors, including historical measurement data, and ``safety'' or dampening factors. In short, the former is the throughput measurement as seen by the end-point, while the latter is the input value to the client's rate adaptation algorithm.

\todo{Instantaneous and smoothed are defined here, but then never used again.}

\begin{figure*}[t!]
  \centering
  \subfloat[DSL]{
    \includegraphics[width=\textwidth]{figures/Throughput_DSL.pdf}
    \label{fig:throughput-clients-DSL}
  }
  \\
  \subfloat[FTTC]{
    \includegraphics[width=\textwidth]{figures/Throughput_FTTC.pdf}
    \label{fig:throughput-clients-FTTC}
  }
  \caption{dash.js client throughput measurements \todo{Times}}
  \label{fig:throughput-clients}
\end{figure*}

Figure~\ref{fig:throughput-clients} shows the measurement consistency obtained by clients with New CWV enabled. For example, in all FTTC scenarios (Figure~\ref{fig:throughput-clients-FTTC}), New CWV has a steeper gradient. This indicates that the throughput estimates are more consistent, falling within a tighter range of values. Since clients are able to leave slow-start earlier, \emph{cwnd} oscillates less when compared to clients using CWV. In addition to being more consistent, clients with New CWV enabled reported estimates that were lower overall. This can be explained by the behaviour of CWV illustrated in Figure~\ref{fig:cwv}: CWV will always reach the maximum link capacity because of its longer slow-start phase. These findings confirm the results reported by Nazir et al.~\cite{Nazir-2014-performance-evaluation-congestion-window-validation-dash-newcwv}, and support our initial hypothesis that the streaming clients measuring throughput will be able to obtain estimates that are more stable.

\begin{figure}[t!]
  \centering
  \includegraphics[width=.45\textwidth]{figures/lost_packets.pdf}
  \caption{Lost Packets DSL \todo{Times} \todo{CWV on left, New CWV on right} \todo{Split into separate plots to allow subfloating as in other figures}}
  \label{fig:lost-packets}
\end{figure}

We illustrate the impact of New CWV on packet loss rates in Figure \ref{fig:lost-packets}. New CWV consistently achieves lower loss rates when compared to CWV, with New CWV connections having packet loss rates that are up to half that of CWV connections. As explained in Section \ref{sec:background}, New CWV exits slow-start earlier, does not overshoot its window, and therefore is able to avoid the loss seen near the end of slow-start that CWV experiences (Figure~\ref{fig:transmission-after-idle}). Nazir et al.~\cite{Nazir-2014-performance-evaluation-congestion-window-validation-dash-newcwv} observed similar loss values both when New CWV is enabled and when it is not; we attribute this to the much larger RTT value that they used.

\todo{Justify the final sentence better}

% To evaluate this hypothesis, we collect two of the client's available bandwidth measurements. The first is the raw estimate obtained by dividing the number of received bytes by time taken to download them. The second estimation, used by the application level adaptation algorithm, is a function of the raw bandwidth estimate, already explained, and additional factors, including historical values. We refer to the former as the \emph{instantaneous} and the latter the \emph{smoothed} estimate.

% Figure~\ref{fig:throughput-clients} shows CDFs of the instantaneous and smoothed bandwidth estimates. Intuitively, steeper lines indicate more consistent estimates: that is, a greater proportion of the estimates fall within a tighter range of values. As shown in Figure~\ref{fig:throughput-clients}, New CWV tends to report bandwidth estimates that are more consistent. This validates both our initial hypothesis, and the results reported by Nazir et al.~\cite{Nazir-2014-performance-evaluation-congestion-window-validation-dash-newcwv}. We also find that New CWV produces bandwidth estimates that are lower overall. This is since when New CWV is enabled, connections will not hit the available capacity cap as fast. However, this pre-emptive mechanism allows for higher consistency on links where nearly all of the available bandwidth is used (FTTC scenarios in Figure \ref{fig:throughput-clients}). \todo{not sure what the last couple of sentences are trying to say}.


% However, our packet loss results (Figure \ref{fig:transmission-after-idle}) differ from those reported by Nazir et al.~\cite{Nazir-2014-performance-evaluation-congestion-window-validation-dash-newcwv}. They observed similar loss values with and without New CWV enabled, while we observe that when disabled, connections lose more packets (at a rate close to 1:10) compared to when enabled. Figure \ref{fig:transmission-after-idle} shows the transfer patterns for a single video chunk after an idle period. It can be seen that when New CWV is disabled the majority of the packets lost are near the end of the slow-start period, when the \texttt{\emph{cwnd}} size allows for transferring data beyond the link's capacity \todo{words, backref to sec 2 diagram}. In contrast, since New CWV enters slow-start with an upper bound of its previous window value, New CWV is able to enter congestion avoidance faster without losing packets as a result of a large \texttt{\emph{cwnd}} value. This explains the higher loss values that we have observed.

%--------------------------------------------------------------------------------------------------
\subsection{Impact on Video QoE}
\label{sec:QoE-impact}

\begin{figure*}
  \centering
  \includegraphics[width=\textwidth, keepaspectratio]{figures/bitrate_derivative_distribution.pdf}
  \caption{Absolute Bitrate Switches \todo{Times} \todo{Flip order of labels in legend: CWV left, New CWV right}}
  \label{fig:bitrate-switches}
\end{figure*}

\begin{figure}
      \includegraphics[width=.45\textwidth, keepaspectratio]{figures/Rebuffer_Ratio.pdf}
    \caption{Rebuffer Ratio \todo{Times} \todo{CWV on left, New CWV on right} \todo{Split into separate plots to allow subfloating as in other figures}}
    \label{fig:rebuffer-ratio}
\end{figure}

To evaluate whether the improved transport layer performance of New CWV translates into improved QoE at the application layer, we report results for the rebuffer ratio and the bitrate switch frequency. We report results for the DSL evaluations, as this link type, in combination with the video encodings used, best highlights the scenario in which New CWV is most beneficial. With FTTP the combination of clients and video encodings enabled all clients to stream at the highest quality without hitting the link limits. This was also the case for all FTTC simulations, with the exception of the 5 client one. In that scenario, we observed a similar pattern as that shown for DSL. We believe that the problem we describe here is applicable to faster links as network demands increase, with higher resolutions being widely used or as other network-heavy applications start competing for the links' resources (e.g., virtual and augmented reality).

% Figure \ref{fig:avg-oscillations} shows average bitrate oscillation. \todo{one/two sentences intuition about the scale}. Bitrate oscillation is typically lower when New CWV is enabled, and this is associated with improved QoE~\todo{\cite{..}}. This is a result of the higher, and less consistent, bandwidth estimates observed when New CWV is not enabled, as described in Section~\ref{sec:transport-impact}. These higher estimates lead to the client requesting chunks at a bitrate that the network cannot ultimately support, leading to increased oscillations and lower QoE. A similar pattern can be seen in Figure~\ref{fig:rebuffer-ratio}, which shows the rebuffer ratio.

Figure \ref{fig:bitrate-switches} shows the observed bitrate switch distribution. The proportion of the requested chunks (\emph{y axis}) is shown using a logarithmic scale. The representation change (\emph{x axis}) is the magnitude of the chunk-by-chunk bit rate switches. For example, if a chunk is requested at the same bit rate as the preceding chunk, the difference is zero. If the chunk is of the next higher encoding, compared to its predecessor, the difference is 1, and if it is of the next lower encoding the difference is -1, and so on. We see that for the case with 5 clients, connections without New CWV experience non-zero quality switches for 17.4\% of the video duration, compared to 5.7\% for connections with New CWV. For a video consisting of just over 200 chunks, an 11.7 percentage point difference means that, on average, New CWV connections see 24 fewer switches over the course of the video playback.

Figure \ref{fig:rebuffer-ratio} shows the percentage that the of the whole video playback that the connection spent in rebuffering state, where the video player stalled, and playback did not progress. Again, looking at the 5 client case, we see that New CWV experiences less rebuffering overall. The mean rebuffering values for the 5 client case are 5\% and 1.5\% for CWV and New CWV respectively. For a 10 minute video, this 3.5 percentage point difference accounts for over 21 seconds of rebuffering time.

% For a 10 minute video, that 6\% difference accounts for 36 seconds shorter video rebuffering for New CWV connections throughout the video playback.

We conclude that New CWV achieves higher video stability when multiple clients are competing on a constrained link, with improved encoding stability and decreased rebuffering time. We have observed this behaviour mainly on the simulated DSL link, since our highest video encoding is just under 5Mbps, however, we note that in practice higher encodings are also used~\cite{online-youtube-encodings}.

%--------------------------------------------------------------------------------------------------
\subsection{Summary}
\label{sec:summary}

We have validated the throughput results observed by Nazir et al.~\cite{Nazir-2014-performance-evaluation-congestion-window-validation-dash-newcwv} (Figure~\ref{fig:throughput-clients}), but also observed higher packet loss where New CWV is not enabled (Figure~\ref{fig:lost-packets}). Furthermore, our results demonstrate that New CWV's more consistent bandwidth measurements (Figure~\ref{fig:throughput-clients}) translate to fewer representation switches (Figure \ref{fig:bitrate-switches}). We also conclude that the more consistent measurements better match the available network conditions, allowing clients using New CWV to better adapt to the constraints of the bottleneck link, and thus to request chunks that can be delivered on time without need of rebuffering (Figure \ref{fig:rebuffer-ratio}).

%In Section \ref{sec:evaluation}, we saw that New CWV improves video stability by sacrificing some rendered picture quality for links with restrained capacities and multiple clients. We also, observed that when the there is sufficient link capacity to satisfy the most demanding representation, as expected the rendered quality is high and there are no stability issues. In other words, if the network resources are oversupplied with regard to the highest video requirement, there are no issues.  However, if the client is able to download at rate over the highest representation requirement, then whether the download happens 20\% or 50\% faster, makes no significant impact. The only real benefit is that the buffers at the client would fill quicker. Therefore, clients can potentially impose a maximum limit to their desired download rate and leave the rest of the link unused for other flows.
% ^ not sure what this adds: might just need rewording

%==================================================================================================
\section{Related Work}
\label{sec:related}

New CWV enhances the CWV algorithm~\cite{rfc2861-2000-padhye-congestion-window-validation}. Both CWV and New CWV attempt to solve the issue of connection resumption after an idle period in which TCP's view of the network has become stale. While CWV addresses the issue for bulk, network-limited applications, New CWV improves the algorithm for rate-limited applications. As a result of their changes, both algorithms alter TCP's send rate. This core idea, of altering TCP's sending dynamics, is not new. Building on work by Mathis et al.~\cite{Mathis-1997-the-macroscopic-behavior-tcp} and Padhye et al.~\cite{Padhye-1998-modelling-tcp-throughput}, there has been a significant number of proposals on TCP friendliness and rate control~\cite{rfc-5348-tfrc,Rossi-2010-ledbat,Arun-2018-copa}, including for multimedia~\cite{Carlucci-2016-Analysis-WebRTC,Choi-2007-fairer-tfrc}.

\todo{Better link TCP friendliness with the storyline}

While the transport layer is ever evolving, adaptation algorithms in the application layer have so far attempted to mask the transport behaviour. As such, three main concepts for adaptation algorithms have been proposed: throughput-based \cite{Sun-2016-cs2p, Jiang-2012-improving-fairness-http-video-festive}, buffer-based \cite{Spiteri-2016-BOLA,Huang-2015-A-buffer-based-approach-to-rate-adaptation-bba}, and hybrid \cite{Spiteri-2019-from-theory-to-practice-sabre,Wang-2016-squad}. Rate-based solutions~\cite{Li-2014-probe-and-adapt-panda,Liu-2011-rate-adaptation} have also been proposed, but these are yet to be accepted by the DASH industry forum~\cite{online-dashif}.

%As different adaptation algorithms were proposed, the need of metrics to evaluate and compare them arose. Early research provided metrics to evaluate the perceived QoE \cite{Cranley-2006-user-perception-adapting-video}. However, later \cite{Balachandran-2012-quest-for-internet-video-qoe} argued that no single combination of low-level metrics can provide a good QoE score for measuring video. Since then, QoE defining research relied on modelling and predictive patterns, however, a single agreed upon metric to measure to QoE has still not been found.

Work on adaptation algorithms has slowed, with most recent proposals optimising for specific use cases (e.g., \cite{Karagkioules-2020-achieving-low-latency}). Partly, this is because of the diminishing returns obtained by increasingly complex algorithms \cite{Yin-2015-a-control-theoritic-approach}. In this work, we identified cases (e.g., multiple clients competing on a constrained link) where the current state-of-the art algorithms perform poorly. We showed that transport changes, that is, enabling New CWV, can have positive QoE impact, with up to 4\% improved rebuffering and 12\% more stable chunk selection \todo{Are these percentage improvements, or percentage \emph{point} improvements?}. We hope to open a discussion and allow more researchers seeking to improve adaptation algorithms to look into adapting the transport layer to better suit video traffic.

We believe that with changes to the transport layer, simple, network-reactive, throughput algorithms will be able to perform comparable to other more complex solutions, such as buffer-based or the dynamic algorithms. \todo{Why is this a good thing?}

%==================================================================================================
\section{Conclusions}
\label{sec:conclusion}

In this paper, we have shown that enabling New CWV improves video playback stability. We compared video delivery with CWV and New CWV, and validated the results shown by previous work~\cite{Nazir-2014-performance-evaluation-congestion-window-validation-dash-newcwv}. We reported video delivery scenarios using emulated links representative of connections within a country or region, and examined scenarios with different numbers of clients. We found that enabling New CWV, a transport layer change, can improve application layer performance, reducing the number of encoding switches by up to 12\% and rebuffering time by up to 4\% \todo{Are these percentage improvements, or percentage \emph{point} improvements?}.

\todo{This is very descriptive: is there a punchier, big-picture take-away?}

Future work might look at the performance of these algorithms under more dynamic environments, for example, if all clients join the session at random times or in the presence of other cross-traffic. 

%==================================================================================================
%\section{Acknowledgements}

% Acknowledge funding sources.

%==================================================================================================
\bibliographystyle{ACM-Reference-Format}
\bibliography{paper}
%==================================================================================================
% The following information gets written into the PDF file information:
\ifpdf
  \pdfinfo{
    /Title        (Does TCP's New Congestion Window Validation Improve HTTP Adaptive Streaming Performance?)
    /Author       (-)
    /Subject      (Video Streaming)
    /Keywords     (TCP, MPEG DASH, Congestion Window Validation)
    /CreationDate (D:20220317130400Z)
    /ModDate      (D:20220317130400Z)
    /Creator      (LaTeX)
    /Producer     (pdfTeX)
  }
  % Suppress unnecessary metadata, to ensure the PDF generated by pdflatex is
  % identical each time it is built. This needs pdfTeX 3.14159265-2.6-1.40.17
  % or later.
  \ifdefined\pdftrailerid
    \pdftrailerid{}
    \pdfsuppressptexinfo=15
  \fi
\fi


%==================================================================================================
\end{document}
% vim: set ts=2 sw=2 tw=75 et ai:
