\documentclass[10pt,sigconf]{acmart}
\synctex=1
\graphicspath{{figures/}{doc/paper/}}
\usepackage[l2tabu,orthodox]{nag}
\usepackage[utf8x]{inputenc}
\usepackage[british]{babel}
\usepackage{microtype}
\usepackage[caption=false]{subfig}
\usepackage{graphicx}
\usepackage{url}
\usepackage{color}

\frenchspacing
\uchyph=0

\newcommand{\todo}[1]{\textbf{\textcolor{red}{To do: #1}}}
\newcommand{\idea}[1]{{\textcolor{blue}{#1}}}

%==================================================================================================
\begin{document}
\title{Does TCP's New Congestion Window Validation Improve HTTP Adaptive Streaming Performance?}

\author{Mihail Yanev}
    \orcid{0000-0002-9814-1017}
    \affiliation{
      \institution{University of Glasgow}
      \streetaddress{School of Computing Science}
      \city{Glasgow}
      \postcode{G12 8QQ}
      \country{UK}
    }
    \email{m.yanev.1@research.gla.ac.uk}

  \author{Stephen McQuistin}
    \orcid{0000-0002-0616-2532}
    \affiliation{
    \institution{University of Glasgow}
    \streetaddress{School of Computing Science}
    \city{Glasgow}
    \postcode{G12 8QQ}
    \country{UK}
    }
    \email{sm@smcquistin.uk}

\author{Colin Perkins}
  \orcid{0000-0002-3404-8964}
  \affiliation{
    \institution{University of Glasgow}
    \streetaddress{School of Computing Science}
    \city{Glasgow}
    \postcode{G12 8QQ}
    \country{UK}
  }
  \email{csp@csperkins.org}

\acmYear{2022}
\copyrightyear{2022}
\setcopyright{acmcopyright}
\acmConference{NOSSDAV '22}{June 17, 2022}{Athlone, Ireland}
\acmPrice{15.00}
% \acmDOI{10.1145/3284850.3284856}
% \acmISBN{978-1-4503-6082-1/18/12}

%==================================================================================================
\begin{abstract}
  % Four sentences:
  %  - State the problem
  %  - Say why it's an interesting problem
  %  - Say what your solution achieves
  %  - Say what follows from your solution

HTTP adaptive streaming video flows exhibit on-off behaviour, with frequent idle periods, which can interact poorly with TCP's congestion control algorithms. New congestion window validation (New CWV) modifies TCP to allow senders to restart more quickly after certain idle periods. While previous work has shown that New CWV can improve \emph{transport} performance for streaming video, it remains to demonstrate that this translates to improved \emph{application} level performance, in terms of playback stability. In this paper, we show that enabling New CWV can reduce video re-buffering events by up to 4\%, and limit representation switches by 12\%, without any changes to existing rate adaptation algorithms.

\end{abstract}
\maketitle
%==================================================================================================
\section{Introduction}
\label{sec:introduction}

% A good paper introduction is fairly formulaic. If you follow a simple set
% of rules, you can write a very good introduction. The following outline can
% be varied. For example, you can use two paragraphs instead of one, or you
% can place more emphasis on one aspect of the intro than another. But in all
% cases, all of the points below need to be covered in an introduction, and
% in most papers, you don't need to cover anything more in an introduction.
%

Video streaming over HTTP is commonplace, and comprises the majority of Internet traffic~\cite{Sandvine-2019-global-internet-report}. Performance of HTTP adaptive streaming is generally good, and gives a high-quality user experience.
  
There are, however, some scenarios where HTTP adaptive streaming performs poorly~\cite{Spiteri-2016-BOLA,Kua-2017-a-survey-rate-adaptation-dash}. In particular, the interaction between the on-off traffic patterns generated by chunked streaming applications and TCP congestion control algorithms can reduce the performance of throughput-based video rate adaptation schemes~\cite{Akhshabi-2012-http-adaptive-players-compete,Stohr-2017-where-are-the-sweet-spots-maci}. In some cases, this is due to TCP's congestion window validation (CWV)~\cite{rfc2861-2000-padhye-congestion-window-validation} algorithm, which, while preventing TCP clients from sending using stale knowledge of the network, has been shown to negatively impact the throughput of rate-limited applications~\cite{Nazir-2014-performance-evaluation-congestion-window-validation-dash-newcwv}, including HTTP adaptive streaming. New congestion window validation (New CWV)~\cite{rfc7661-2015-fairhurst-new-cwnd-validation} has been proposed to address this. Prior work~\cite{Nazir-2014-performance-evaluation-congestion-window-validation-dash-newcwv} has demonstrated that New CWV has the desired \emph{transport} layer impact, but it remains to show that this translates to improved quality of experience (QoE) performance at the \emph{application} layer. This is not guaranteed, given the complexity that exists at both layers, and that results from their interaction. For example, large discrepancies between the video's bandwidth requirements and the available link capacity, or the requirement for stable, long-lived connections in modern streaming video players (e.g., dash.js~\cite{online-dashjs}), can influence rate adaptation~\cite{Spiteri-2019-from-theory-to-practice-sabre}.

% Paragraph 1: Motivation. At a high level, what is the problem area you
% are working in and why is it important? It is important to set the larger
% context here. Why is the problem of interest and importance to the larger
% community?


% Paragraph 2: What is the specific problem considered in this paper? This
% paragraph narrows down the topic area of the paper. In the first
% paragraph you have established general context and importance. Here you
% establish specific context and background.



% Paragraph 3: "In this paper, we show that...". This is the key paragraph
% in the introduction - you summarize, in one paragraph, what are the main
% contributions of your paper, given the context established in paragraphs
% 1 and 2. What's the general approach taken? Why are the specific results
% significant? The story is not what you did, but rather:
%  - what you show, new ideas, new insights
%  - why interesting, important?
% State your contributions: these drive the entire paper.  Contributions
% should be refutable claims, not vague generic statements.

In this paper, we investigate whether enabling New CWV improves video playback stability, and more generally, improves video QoE. To test our hypothesis, we compare two video streams using TCP New Reno, one with CWV and with New CWV. We collect standard video performance metrics, including bit-rate oscillation, and stall time, to measure stability and QoE. Further, to quantify the impact of New CWV with respect to the inferred network state at the client, we also record the immediate and smoothened client's current link capacity estimations for each delivered video chunk.

In particular, we make the following contributions: (i) an implementation of New CWV for Linux (kernel version 5.4)~\footnote{The code used in this paper will be made available with the camera-ready version.}; (ii) a testbed setup for evaluating New CWV's application layer impact; and (iii) results that demonstrate that New CWV improves video stability, with a 12\% reduction in bit rate switches, and a 4\% reduction in rebuffering time.


To the best of our knowledge, this is the first paper that studies New CWV's application layer impact. Nazir et al. \cite{Nazir-2014-performance-evaluation-congestion-window-validation-dash-newcwv} demonstrated New CWV's effect on the transport layer: we validate their results in \S\ref{sec:transport-impact}. There has been a large amount of work that has proposed new application layer rate adaptation algorithms~\cite{Mok-2012-qdash,Huang-2015-A-buffer-based-approach-to-rate-adaptation-bba, Yin-2015-a-control-theoritic-approach}. In contrast, we only change the transport algorithm and leave the application as is, studying the transport's impact on the application. Improving performance via transport layer modifications could allow for simpler rate adaptation algorithms at the application layer.


We structure the remainder of this paper as follows. In Section~\ref{sec:background}, we introduce TCP congestion window validation, including its limitations with respect to HTTP adaptive video flows, before describing New CWV. Section~\ref{sec:evaluation} describes our experimental setup, and the transport and application layer impact of enabling New CWV. Section~\ref{sec:related} describes related work, and Section~\ref{sec:conclusion} concludes.

%==================================================================================================
\section{Congestion Window Validation}
\label{sec:background}

\begin{figure}
  \centering
    \subfloat[CWV]{
      \includegraphics[width=.23\textwidth]{figures/cwv.pdf}
      \label{fig:cwv}
    }
    \subfloat[New CWV]{
      \includegraphics[width=.23\textwidth,]{figures/new_cwv.pdf}
      \label{fig:newcwv}
    }
    \caption{Illustration of \emph{cwnd} growth following an idle period}
    \label{fig:cwnd-growth-after-idle}
\end{figure}

In HTTP adaptive streaming, a server provides pre-encoded video chunks in different representations, each encoded at multiple bit rates, while the client, using a rate adaptation algorithm, determines the best representation to request at any given time. The goal of the client is to maximise QoE within the network's capacity. This can be a challenge since different, often contradictory, QoE heuristics need be considered simultaneously~\cite{Seufert-2015-A-Survey-on-QoE-Dash}. 

Throughput-based rate adaptation algorithms for HTTP adaptive streaming use an estimate of the current network conditions to determine the representation that should be requested. These algorithms require a stable and accurate throughput estimate in order to perform well. However, the interaction between the on-off traffic pattern of streaming video and TCP's congestion control algorithm can lead to significant fluctuations in throughput, impacting the performance of throughput-based rate adaptation algorithms.


In particular, during idle periods in between transmission of video chunks, the TCP congestion controller's knowledge of the network capacity becomes stale. To avoid sending with a possibly unrepresentative congestion window, once the link has been idle for a period longer than the connection's retransmission timeout, $T_{rto}$, the congestion window validation~\cite{rfc2861-2000-padhye-congestion-window-validation} (CWV) algorithm resets the TCP congestion window (\emph{cwnd}) to its initial value and forces the connection to re-enter slow-start after an idle period. Figure~\ref{fig:cwv} illustrates the behaviour of CWV following an idle period. CWV has become standard practice~\cite{rfc5681-congeston-control}, and is enabled by default in the latest stable Linux kernel (5.4). However, re-entering slow-start like this, results in packet loss once the CWND grows beyond the link capacity (Figure~\ref{fig:transmission-after-idle-reno}). This was found to interact poorly with HTTP adaptive streaming and other application-limited transmissions~\cite{Esteban-2012-Interactions-HTTP-TCP}.

To address the slow-start re-entering issue, new congestion window validation~\cite{rfc7661-2015-fairhurst-new-cwnd-validation} has been proposed. One of the modifications in New CWV is that rather than relying on slow-start until packet loss to re-discover an appropriate \emph{cwnd} value after an idle period, New CWV preserves the \emph{cwnd} before the idle period as its slow-start threshold (\emph{ssthresh}), i.e., it uses that value to later exit the slow-start phase. Figure~\ref{fig:newcwv} shows the growth of \emph{cwnd} following an idle period under New CWV.

\begin{figure}[t!]
  \centering
  \subfloat[CWV]{
    \includegraphics[width=.45\textwidth, keepaspectratio]{figures/lost_packets_vreno.pdf}
    \label{fig:transmission-after-idle-reno}
  }
  \\
    \subfloat[New CWV]{
      \includegraphics[width=.45\textwidth, keepaspectratio]{figures/lost_packets_newcwv.pdf}
      \label{fig:transmission-after-idle-newcwv}
    }
    \caption{Resumption after an idle period}
    \label{fig:transmission-after-idle}
\end{figure}

To evaluate the transport performance of New CWV, we have implemented the algorithm within the Linux kernel. We used \cite{online-newcwv-base} as a base, which provides a Kernel 3.18 implementation. Our implementation altered two files adding 143 and removing 49 lines of code. We use this implementation to better illustrated the impact of New CWV on flows restarting after an idle period, as shown in Figure~\ref{fig:transmission-after-idle}.
As shown, the connection using New CWV uses the previously set \emph{ssthresh} value and leaves slow-start early. This results in New CWV connections not experiencing any packet loss (Figure~\ref{fig:transmission-after-idle-newcwv}), after reaching their set \emph{ssthresh} value in the third flight of packets after restarting. In contrast, if the same connection used CWV, the senders would not have preserved the \emph{ssthresh} value that way and would rely on loss to exit slow-start, as seen at the end of the third and fourth flights of packets in Figure~\ref{fig:transmission-after-idle-reno}. In the presented case, CWV enters congestion avoidance around 160 milliseconds after the transmission restart (Figure \ref{fig:transmission-after-idle-reno}), while New CWV is able to enter congestion avoidance around the 110th millisecond mark (Figure \ref{fig:transmission-after-idle-newcwv}).

Overall, New CWV results in fewer lost packets, and returns to its previous sending rate without overshoot after loss, giving more predictable transmission.

New CWV has previously been shown to improve the \emph{transport} layer performance of rate-limited applications when compared with CWV~\cite{Nazir-2014-performance-evaluation-congestion-window-validation-dash-newcwv}, and our implementation and the results we have presented here validate that. It remains to show how this translates into \emph{application} layer performance, particularly since this is not guaranteed~\cite{Spiteri-2016-BOLA}. In Section~\ref{sec:evaluation}, we first validate the results of Nazir et al.~\cite{Nazir-2014-performance-evaluation-congestion-window-validation-dash-newcwv}, before testing our hypothesis that New CWV will enable applications to obtain more consistent throughput estimates, and, in turn, improve the stability of throughput-based rate adaptation algorithms.

%==================================================================================================
\section{Evaluating New CWV for Video}
\label{sec:evaluation}

\begin{figure}
  \centering
  \includegraphics[width=.5\textwidth]{figures/setup.pdf}
  \caption{Experimental Setup}
  \label{fig:experimental-setup}
\end{figure}

We first describe our experimental setup (\S\ref{sec:experimental-setup}), which we use to investigate the transport layer impact, verify whether New CWV connections obtain more consistent throughput estimates (\S\ref{sec:transport-impact}), and later to also investigate the application impact, and more specifically, the difference on video QoE that New CWV connections observe (\S\ref{sec:QoE-impact}). Finally, we summarise our findings (\S\ref{sec:summary}).

%--------------------------------------------------------------------------------------------------
\subsection{Experimental Setup}
\label{sec:experimental-setup}

% Overview
Our evaluation testbed consists of a network emulated in Mininet, running on Ubuntu 20.04, as shown in Figure~\ref{fig:experimental-setup}. Both the server and its clients use TCP New Reno, widely used for video delivery~\cite{Mishra-2019-the-great-internet-tcp-congestion-control-census}. In addition, both are running a modified Linux Kernel (5.4.0). The modifications include a version of New CWV ported to that kernel, alongside RFC 3339~\cite{rfc3339-precise-timestamps} compliant timestamps, to enable better event tracking with higher timing precision. \texttt{tcpdump} is used to allow network activity to be reconstructed, which we use to study the packet loss.

% Server
The server uses \texttt{nginx} (version 1.18) with HTTP/2 delivery enabled. The server provides three representations of Big Buck Bunny~\cite{online-bbb}, encoded at 480p (requiring bandwidth of 0.44Mbps), 720p (2.64Mbps), and 1080p (4.82Mbps). Each representation is provided in chunks that are 3 seconds in duration.

 %  Client
Each client uses Firefox (version 91) with the dash.js (version 4.0.0) player. While the current state-of-the-art rate adaptation algorithm is DYNAMIC~\cite{Spiteri-2019-from-theory-to-practice-sabre}, we opt to use the throughput algorithm for our experiments since we believe it would better showcase the effect of enabling New CWV. This is since DYNAMIC combines the throughput algorithm with an enhanced version of the BOLA algorithm~\cite{Spiteri-2016-BOLA}. For completeness and to make a more accurate case for the technologies currently in use, we also report the results of ABR DYNAMIC.

The network is configured with a bottleneck RTT of 40ms, a reasonable RTT value to the closest video CDN replica. The routers' queues are sized to the bandwidth delay product. Three different bandwidth profiles are evaluated, representing DSLv2 (10Mbps), FTTC (50Mbps), and FTTP (145Mbps) links; e.g., as are typical in the UK~\cite{online-ofcom-report}. Below we show the results for the DSLv2 and FTTC links. However, as higher resolution video and other network-heavy operations such as virtual reality environments become more available we expect the issues observed could translate to the links with higher capacity (i.e. FTTC and FTTP).

As our experiments investigate the effect of New CWV when only video traffic is present, the chosen homogenous RTT value fits our experiment design, since in practice, traffic for such scenarios would likely come from the same streaming server or CDN replica. Furthermore, we note that a set of experiments that evaluate New CWV in a more realistic scenario (e.g., with cross traffic, interacting with multiple congestion control algorithms, and different access link types) would be useful but is beyond the scope of this paper.

To evaluate the impact of congestion and competing flows, each simulation was run with multiple clients (1, 2, 3, and 5 clients) simultaneously requesting video. Finally, to reduce noise, we ran each combination of CWV or New CWV, number of clients, and link type, 10 times before reporting the average results. The results presented includes data accumulated from 240 simulations (\emph{2 algorithms $\times$ 3 link types $\times$ 4 client variations $\times$ 10 repetitions}). 

During each run we collect the client's bandwidth estimations. Additionally, to evaluate the video QoE impact we collect information to report the rebuffer ratio and the bitrate switch frequency distribution.

%--------------------------------------------------------------------------------------------------
\subsection{Impact on Transport Performance} 
\label{sec:transport-impact}

New CWV alters TCP's \emph{cwnd} sizing behaviour, allowing it to recover more quickly after an idle period in an active TCP connection. As shown in Figures~\ref{fig:newcwv} and \ref{fig:transmission-after-idle-newcwv}, New CWV avoids the packet loss associated with CWV, and we therefore expect clients to report more stable available link bandwidth estimates. 

To evaluate this hypothesis, we collect the client's ``instantaneous'' and ``smoothed'' bandwidth estimates. The ``instantaneous'' estimate is obtained by dividing the size of the chunk, in bytes, by the time taken to download it. The ``smoothed'' estimate takes the ``instantaneous'' estimate, but combines it with other factors, including historical measurement data, and ``safety'' or dampening factors. In short, the former is the throughput measurement as seen by the end-point, while the latter is the input value to the client's rate adaptation algorithm.

\begin{figure*}[t!]
  \centering
  \subfloat[DSL]{
    \includegraphics[width=\textwidth]{figures/Throughput_DSL.pdf}
    \label{fig:throughput-clients-DSL}
  }
  \\
  \subfloat[FTTC]{
    \includegraphics[width=\textwidth]{figures/Throughput_FTTC.pdf}
    \label{fig:throughput-clients-FTTC}
  }
  \caption{dash.js client throughput measurements}
  \label{fig:throughput-clients}
\end{figure*}

Figure~\ref{fig:throughput-clients} shows the cumulative distribution function of all instantaneous and smoothed throughput measurements as seen by the clients. In it, a steep function would indicate that most throughput values have seen little variation, and are therefore highly consistent. Most FTTC cases, do showcase a steeper function for connections using New CWV, compared connections not using confirming our hypothesis. 

The case where, we do not see this pattern for FTTC is the one client on FTTC link. We attribute the higher variance of New CWV in this case to the fact that the highest video requirement was much lower than the network capacity, and therefore the New CWV client could not operate close to the link's capacity to get more accurate throughput measurements. For all other cases, since clients are able to leave slow-start earlier, \emph{cwnd} oscillates less when compared to that of clients using CWV. In addition to being more consistent, clients with New CWV enabled reported estimates that were lower overall. This can be explained by the behaviour of CWV illustrated in Figure~\ref{fig:cwv}: CWV will always reach the maximum link capacity because of its longer slow-start phase. These findings confirm the results reported by Nazir et al.~\cite{Nazir-2014-performance-evaluation-congestion-window-validation-dash-newcwv}, and support our initial hypothesis that the streaming clients measuring throughput will be able to obtain estimates that are more stable. Finally, we have omitted the FTTP results from Figure~\ref{fig:throughput-clients}, since in all cases, the link capacity was much higher than highest video encoding requirement and in all these scenarios, both connections with and without New CWV showcased similar throughput measurements, lower than the true available link capacity. In all such cases, both algorithms have failed to assess the true available capacity since the size of the sent video chunks was not enough to create a large enough CWND to saturate the link.

\begin{figure}[t!]
  \centering
  \includegraphics[width=.45\textwidth]{figures/lost_packets.pdf}
  \caption{Lost Packets DSL}
  \label{fig:lost-packets}
\end{figure}

We illustrate the impact of New CWV on packet loss rates in Figure \ref{fig:lost-packets}. New CWV consistently achieves lower loss rates when compared to CWV, with New CWV connections having packet loss rates that are up to half that of CWV connections. As explained in Section \ref{sec:background}, New CWV exits slow-start earlier, does not overshoot its window, and therefore is able to avoid the loss seen near the end of slow-start that CWV experiences (Figure~\ref{fig:transmission-after-idle}). Nazir et al.~\cite{Nazir-2014-performance-evaluation-congestion-window-validation-dash-newcwv} observed similar loss values both when New CWV is enabled and when it is not; we believe this due to the much larger RTT value they used.


%--------------------------------------------------------------------------------------------------
\subsection{Impact on Video QoE}
\label{sec:QoE-impact}

\begin{figure*}
  \centering
  \includegraphics[width=\textwidth, keepaspectratio]{figures/bitrate_derivative_distribution.pdf}
  \caption{Absolute Bitrate Switches (ABR Throughput)}
  \label{fig:bitrate-switches}
\end{figure*}

\begin{figure*}
  \centering
  \includegraphics[width=\textwidth, keepaspectratio]{figures/bitrate_derivative_distribution_dynamic.pdf}
  \caption{Absolute Bitrate Switches (ABR Dynamic)}
  \label{fig:bitrate-switches-dynamic}
\end{figure*}


\begin{figure}
      \includegraphics[width=.45\textwidth, keepaspectratio]{figures/Rebuffer_Ratio.pdf}
    \caption{Rebuffer Ratio (ABR Throughput)}
    \label{fig:rebuffer-ratio}
\end{figure}

\begin{figure}
  \includegraphics[width=.45\textwidth, keepaspectratio]{figures/Rebuffer_Ratio_dynamic.pdf}
\caption{Rebuffer Ratio (ABR Dynamic)}
\label{fig:rebuffer-ratio-dynamic}
\end{figure}

To evaluate whether the improved transport layer performance of New CWV translates into improved QoE at the application layer, we report results for the rebuffer ratio and the bitrate switch frequency. We report results for the DSL evaluations, as this link type, in combination with the video encodings used, best highlights the scenario in which New CWV is most beneficial. With FTTP the combination of clients and video encodings enabled all clients to stream at the highest quality without hitting the link limits. This was also the case for all FTTC simulations, with the exception of the 5 client one. In it, we observed a similar pattern as that shown for DSL. We believe that the problem we describe here would be applicable to faster links as applications' network demands increase, f.e., higher video resolutions are seeing more deployment, or as other network-heavy applications start competing for the links' resources (e.g., virtual and augmented reality). Furthermore, we note that the issues we found with the DSL link still have high impact as over 30\% of the households in the UK used a DSLv2 connection or similar in 2019 \cite{online-ofcom-report}. 

\todo{Make the DSL link case in the last sentence more global and not as UK specific.}

Figure \ref{fig:bitrate-switches} shows the observed bitrate switch distribution using the ABR Throughput algorithm. The proportion of the requested chunks (\emph{y axis}) is shown using a logarithmic scale. The representation change (\emph{x axis}) is the magnitude of the chunk-by-chunk bit rate switches. For example, if a chunk is requested at the same bit rate as the preceding chunk, the difference is zero. If the chunk is of the next higher encoding, compared to its predecessor, the difference is 1, and if it is of the next lower encoding the difference is -1, and so on. We see that for the case with 5 clients, connections without New CWV experience non-zero quality switches for 17.4\% of the video duration, compared to 5.7\% for connections with New CWV. For a video consisting of just over 200 chunks, an 11.7 percentage point difference means that, on average, New CWV connections see 24 fewer switches over the course of the video playback.

Figure \ref{fig:bitrate-switches-dynamic} shows the observed bitrate switch distribution with the DYNAMIC adaptive bit-rate algorithm (ABR). We see that while the stability for connections without New CWV has improved, for connections where New CWV was enabled, their performance was relatively the same as the one shown in Figure \ref{fig:bitrate-switches}. Furthermore, the performance of connections without New CWV is now closer in terms of video stability to the performance of the New CWV connections, however, even with ABR Dynamic enabled, connections without New CWV still showcase worse stability compared to the New CWV connections.

Figure \ref{fig:rebuffer-ratio} shows the fraction of the video playback where the connection's media did not progress or otherwise was waiting for more content before it can resume playing, when using the ABR Throughput algorithm. The value is shown in percentage relative to the whole media playtime. Again, looking at the 5 client case, we see that New CWV experiences less rebuffering overall. The mean rebuffering values for the 5 client case are 5\% and 1.5\% for CWV and New CWV respectively. For a 10 minute video, this 3.5 percentage point difference accounts for over 21 seconds of rebuffering time.

Figure \ref{fig:rebuffer-ratio-dynamic} shows the fraction of the video playback where the connection's media did not progress or otherwise was waiting for more content before it can resume playing for the Dynamic ABR. The value is shown in percentage relative to the whole media playtime. Similar to our discussion on Figure \ref{fig:bitrate-switches-dynamic}, we see that the performance of the connections without New CWV has improved when DYNAMIC is now closer to the performance of the connections where New CWV is enabled. Again, we see that the performance of the New CWV connections does not differ significantly from what was already observed with ABR throughput and is still better than the performance of connections without New CWV regardless of the ABR algorithm choice.


In this paper, we have not shown average bit-rate data, since in all evaluated scenarios connections with and without New CWV have shown similar performance. Additionally, in the rare cases where performance differed, connections without New CWV, had achieved slightly higher average bit-rate, however, this rate was not supported by the network link, as we not that the bit-rate switches and the rebuffering for these connections was also higher.

We conclude that New CWV achieves higher video stability when multiple clients are competing on a constrained link, with improved encoding stability and decreased rebuffering time. We have observed this behaviour mainly on the simulated DSL link, since our highest video encoding is just under 5Mbps, however, we note that in practice higher encodings are also used~\cite{online-youtube-encodings}.

Additionally, we have observed that when New CWV is enabled, the application metrics achieved for connections using either ABR DYNAMIC and ABR Throughput are very similar. We observe that a more complex algorithm, such as ABR Throughput has a significant impact on connections without New CWV. However, even with an increasingly more complex algorithm at the application layer, the QoE of the connections where New CWV is not enabled is still lower than the connections with New CWV.

We found that higher video QoE can be achieved by both deploying more complex application algorithms, or by modifying the underlying transport (e.g., Introducing New CWV). We observed that with improved transport performance, the choice of an application ABR algorithm does not make a significant difference. Also, the reported QoE for connections with an updated transport were higher than the QoE of connections without the transport modification and with the best state-of-the-art ABR algorithm. We conclude that by deploying changes to the transport layer, designing and deploying future ABR algorithms could be easier, as programmers do not need to introduce complex concepts that mask the transport behaviour and can focus on the features of ABR algorithms alone.

%--------------------------------------------------------------------------------------------------
\subsection{Summary}
\label{sec:summary}

We have validated the throughput values reported by Nazir et al.~\cite{Nazir-2014-performance-evaluation-congestion-window-validation-dash-newcwv} (Figure~\ref{fig:throughput-clients}). 
\todo{Clarify how Figure~\ref{fig:throughput-clients} validates Nazir et al}
However, we observed higher packet loss in the cases where New CWV is not enabled (Figure~\ref{fig:lost-packets}). 
\todo{Is `however' correct in the previous, it's implying an opposition to the results of Nazir et al that I'm not sure is warranted}
\todo{Would a clearer phrasing be `The use of New CWV effectively prevents overshoot in slow start (Fig.\ref{fig:transmission after idle}) resulting in lower packet loss rates (Fig. \ref{fig:lost-packet})'}
Furthermore, our results demonstrate that New CWV's more consistent bandwidth measurements (Figure~\ref{fig:throughput-clients}) translate to fewer representation switches (Figure \ref{fig:bitrate-switches}). 
\todo{Not clear that `furthermore' is right, because it's not a direct follow-on.}
We also conclude that the more consistent measurements better match the available network conditions, allowing clients using New CWV to better adapt to the constraints of the bottleneck link, and thus to request chunks that can be delivered on time without need of rebuffering (Figure \ref{fig:rebuffer-ratio}).
\todo{Maybe a more neutral phrasing for this paragraph would be: ``We have presented results as follows: (i) we validated...; (ii) the use of New CWV effectively...; (iii) the more consistent application bandwidth measurements resulting from the use of New CWV...; and (iv) more consistent measurements...''}

The scenarios investigated by this work have used links that emulated a wired
residential Internet connection.  WiFi and cellular links, however, are known
to have very different properties that can affect TCP performance. Understanding
how use of TCP New CWV affects video performance on such links is outside the
scope of this current work.

Our experiments investigated the effects of New CWV on TCP New Reno. This was done to align with
\cite{Nazir-2014-performance-evaluation-congestion-window-validation-dash-newcwv},
and because a recent study \cite{Mishra-2019-the-great-internet-tcp-congestion-control-census}
found New Reno was still widely used for video delivery.
We note that NewCWV could apply for other window-controlled loss based
algorithms, such as TCP CUBIC for example. 

 However, the modifications to TCP that New CWV introduces are a new mechanism to count the unacknowledged packets and an additional exit condition for the slow-start phase. Therefore, these changes should be applicable to CUBIC as well. The main concern that both New CWV and CUBIC alter the slow-start behaviour could be overcome by only increasing the CWND value up to the more conservative option from New CWV and CUBIC.
\todo{TCP Cubic tunes the window growth in slow start after idle such that
the window growth slows as it approaches the previously achieved window,
no? This seems like it would achieve a similar effect to NewCWV. May need
more discussion.}.

 Furthermore, with our findings of New CWV for New Reno and potentially CUBIC hosts 30\% of all Internet video servers could be impacted by this research, as we note that congestion window validation is currently enabled by default in the standard TCP 5.4 Linux Stack. 


%==================================================================================================
\section{Related Work}
\label{sec:related}

New CWV enhances the CWV algorithm~\cite{rfc2861-2000-padhye-congestion-window-validation}. Both CWV and New CWV attempt to solve the issue of connection resumption after an idle period in which TCP's view of the network has become stale. While CWV addresses the issue for bulk, network-limited applications, New CWV improves the algorithm for rate-limited applications. As a result of their changes, both algorithms alter TCP's send rate. Altering the send-rate of TCP should be taken with caution, such that it does not send too fast to cause congestion collapse \cite{Jacobson-1988-congestion-avoidance} but also, to not let other TCP flows have significant effect on their operation (TCP friendliness). The core idea, of altering TCP's sending dynamics, is not new. Building on work by Mathis et al.~\cite{Mathis-1997-the-macroscopic-behavior-tcp} and Padhye et al.~\cite{Padhye-1998-modelling-tcp-throughput}, there has been a significant number of proposals on TCP friendliness and rate control~\cite{rfc-5348-tfrc,Rossi-2010-ledbat,Arun-2018-copa}, including for multimedia~\cite{Carlucci-2016-Analysis-WebRTC,Choi-2007-fairer-tfrc}. These proposals show that it is possible to alter the transport's sending dynamics while remaining friendly to other Internet traffic.

While the transport layer is ever evolving, adaptation algorithms in the application layer have so far attempted to mask the transport behaviour. As such, three main concepts for adaptation algorithms have been proposed: throughput-based \cite{Sun-2016-cs2p, Jiang-2012-improving-fairness-http-video-festive}, buffer-based \cite{Spiteri-2016-BOLA,Huang-2015-A-buffer-based-approach-to-rate-adaptation-bba}, and hybrid \cite{Spiteri-2019-from-theory-to-practice-sabre,Wang-2016-squad}. Rate-based solutions~\cite{Li-2014-probe-and-adapt-panda,Liu-2011-rate-adaptation} have also been proposed, but these are yet to be accepted by the DASH industry forum~\cite{online-dashif}.


Work on adaptation algorithms has slowed, with most recent proposals optimising for specific use cases (e.g., \cite{Karagkioules-2020-achieving-low-latency}). Partly, this is because of the diminishing returns obtained by increasingly complex algorithms \cite{Yin-2015-a-control-theoritic-approach}. In this work, we identified cases (e.g., multiple clients competing on a constrained link) where the current state-of-the art algorithms perform poorly. We showed that transport changes, e.g., enabling New CWV, can have positive QoE impact, with up to 4\% points of improved rebuffering and 12\% points of more stable chunk selection. We hope to open a discussion and allow more researchers seeking to improve adaptation algorithms to look into adapting the transport layer to better suit video traffic.

We believe that with changes to the transport layer, simple, network-reactive, throughput algorithms will be able to perform comparable to other more complex solutions, such as buffer-based or the dynamic algorithms. In turn, this might enable new work in the field to focus on other aspects improving the adaptation process and not to try and mask the transport's behaviour.

%==================================================================================================
\section{Conclusions}
\label{sec:conclusion}

In this paper, we have shown that enabling New CWV improves video playback stability. We compared video delivery with CWV and New CWV, and validated the results shown by previous work~\cite{Nazir-2014-performance-evaluation-congestion-window-validation-dash-newcwv}. We reported video delivery scenarios using emulated links representative of connections within a country or region, and examined scenarios with different numbers of clients. We found that enabling New CWV, a transport layer change, can improve application layer performance, reducing the number of encoding switches by up to 12\% points and rebuffering time by up to 4\% points. To sum up, we have shown that transport changes are able to improve the application QoE. We have also shown that these transport changes make the transmission more predictable. We believe, that these improvements would allow simple throughput algorithms to compete with the current state of the art that attempts to mask the transport behaviour. In turn, this might remove the need for adaptation algorithm implementers to mask the transport behaviour, and could instead allow them to focus on other aspects of the adaptation process. 

Future work might look at the performance of these algorithms under more dynamic environments, for example, if all clients join the session at random times or in the presence of other cross-traffic.

%==================================================================================================
%\section{Acknowledgements}

% Acknowledge funding sources.

%==================================================================================================
\bibliographystyle{ACM-Reference-Format}
\bibliography{paper}
%==================================================================================================
% The following information gets written into the PDF file information:
\ifpdf
  \pdfinfo{
    /Title        (Does TCP's New Congestion Window Validation Improve HTTP Adaptive Streaming Performance?)
    /Author       (-)
    /Subject      (Video Streaming)
    /Keywords     (TCP, MPEG DASH, Congestion Window Validation)
    /CreationDate (D:20220317130400Z)
    /ModDate      (D:20220317130400Z)
    /Creator      (LaTeX)
    /Producer     (pdfTeX)
  }
  % Suppress unnecessary metadata, to ensure the PDF generated by pdflatex is
  % identical each time it is built. This needs pdfTeX 3.14159265-2.6-1.40.17
  % or later.
  \ifdefined\pdftrailerid
    \pdftrailerid{}
    \pdfsuppressptexinfo=15
  \fi
\fi


%==================================================================================================
\end{document}
% vim: set ts=2 sw=2 tw=75 et ai:
